...bietet sich der BDD-Ansatz besonders f�r die Verwendung in einem Liefersystem nach dem Konzept von Continuous Delivery. Dabei handelt es sich um ein Verfahren, welches es erm�glichen soll, Software in sehr kurzen Abst�nden und automatisch in ein Produktivsystem auszuliefern. Kern dieses Verfahrens ist Automatisierung. Ein derartiges Liefersystem kommt besonders im Umfeld von Web-2.0 Anwendungen in Verbindung mit auf Virtualisierung basierenden Verfahren und Technologien. Grunds�tzlich k�nnen dabei aber auch andere Systemtypen ein derartiges Verfahren nutzen, die Ideen hierzu stammen aber vornehmlich aus diesem Umfeld. Dem Konzept zu Folge wird mit einer �nderung der Quellcode-Basis in einem Versionskontrollsystem eine neue Instanz der Delivery-Pipeline erzeugt. Grundlegend basiert der erste Teil der Delivery-Pipeline auf Continuous Integration. Im Umfeld von Java kann z.B. mit Buildtools wie z.B. Ant  oder Maven der Quellcode kompiliert werden und in Zusammenhang mit Komponententests die technische Korrektheit des Systems bewiesen werden. Die Auslieferung des Systems in die Produktivumgebung bedingt die Werkabnahme. Dies impliziert die Erf�llung aller funktionalen und nicht-funktionalen Anforderungen, der Akzeptanzkriterien. BDD stellt hier die notwendige Voraussetzung dar, Akzeptanzkriterien fr�hestm�glich in den Lieferprozess zu implementieren. Jedes �nderung an der Quellcode-Basis f�hrt zu einer m�glichen Auslieferung. Die Erf�llung der Akzeptanzkriterien sind Qualit�tsschranken die das Produktivsystem sch�tzen eine unreife Version auszurollen. Besonders im modernen IT-Betrieb mit immer k�rzeren Entwicklungs- und Release-Zyklen tr�gt ein derartiges Vorgehen zu Verbesserung der wirtschaftlichen Ausnutzung von neuen Funktionalit�ten des entwickelten Systems bei. Die h�heren Aufw�nde die bei der Erstellung von automatisierten Abnahmetests anfallen relativieren sich durch die Chance neue Funktionalit�ten einer Anwendung fr�hestm�glich dem Anwender bereitzustellen. Zudem f�hrt dieser Prozess in Verbindung mit BDD zu einer kontinuierlichen Qualit�tssicherung im gesamten Entwicklungszeitraum die auch f�r ein schnelles Hot-Fix der laufenden Anwendung die notwendigen Sicherungsma�nahmen bereitstellt.